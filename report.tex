\documentclass[11pt,fleqn]{article}

\usepackage{a4wide}

\usepackage{amssymb}
\usepackage{amsfonts}
\usepackage{amsmath}
\usepackage{amsthm}
\usepackage{psfrag}
\usepackage{color}
\usepackage[dvips]{graphicx}
\usepackage{hyperref}
%\usepackage[widespace]{fourier}
%\usepackage{txfonts}
\usepackage{times}
%\usepackage{mathfont}
\usepackage{subcaption}
\usepackage{framed}
\usepackage{svg}
\usepackage{minted}

\begin{document}
	\begin{center}
		{\Large EE2T21 Data Communications Networking - 2024\\[0.1em]
			Bonus Assignment \# 2 \\}
	\end{center}

    \parbox[l][17mm][t]{\textwidth}{Names: Sjoerd Terlouw \hspace{6.66cm}
			Student ID: 5852455}
	\section{Abstract}
    Implementation of the Dijkstra algorithm in Python and performance analysis using the ER random graph.

    \section{Example}
    The following plot gives the shortest path (indicated in red) of a certain graph G using the Dijkstra algorithm. The plot itself is done using the NetworkX python library. 
    The inputs are the edges between the nodes with their weights and the starting and ending node. In the case of figure \ref{fig:example} the starting node is 0 and the ending node is 1.
    The exact format of the input can be found in the code, found in section \ref{section:code}.
    \begin{figure}[H]
        \centering
        \includesvg[width = 0.7\textwidth]{Graph.svg}
        \caption{Visualization of the shortest path found by the Dijkstra algorithm - output of Dijkstra.run\_example()}
        \label{fig:example}
    \end{figure}

    \section{Performance analysis}
    
    \section{Code}
    \label{section:code}
    The code is also provided seperately and is on \href{https://github.com/SjdTl/DCN-Bonus-Assignment-2.git}{\color{blue}github}.
    \inputminted[breaklines=true]{python3}{Dijkstra.py}
\end{document}