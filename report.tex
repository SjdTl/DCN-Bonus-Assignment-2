\documentclass[11pt,fleqn]{article}

\usepackage{a4wide}

\usepackage{amssymb}
\usepackage{amsfonts}
\usepackage{amsmath}
\usepackage{amsthm}
\usepackage{psfrag}
\usepackage{color}
\usepackage[dvips]{graphicx}
\usepackage{hyperref}
%\usepackage[widespace]{fourier}
%\usepackage{txfonts}
\usepackage{times}
%\usepackage{mathfont}
\usepackage{subcaption}
\usepackage{framed}
\usepackage{svg}
\usepackage{minted}

\begin{document}
	\begin{center}
		{\Large EE2T21 Data Communications Networking - 2024\\[0.1em]
			Bonus Assignment \# 2 \\}
	\end{center}

    \parbox[l][17mm][t]{\textwidth}{Names: Sjoerd Terlouw \hspace{6.66cm}
			Student ID: 5852455}
	\section{Abstract}
    Implementation of the Dijkstra algorithm in Python and performance analysis using the ER random graph.

    \section{Example}
    The following plot gives the shortest path (indicated in red) of a certain graph G using the Dijkstra algorithm. The plot itself is done using the NetworkX python library. 
    The inputs are the edges between the nodes with their weights and the starting and ending node. In the case of figure \ref{fig:example} the starting node is 0 and the ending node is 1.
    The exact format of the input can be found in the code, found in section \ref{section:code}.
    \begin{figure}[H]
        \centering
        \includesvg[width = 0.7\textwidth]{Graph.svg}
        \caption{Visualization of the shortest path found by the Dijkstra algorithm - output of Dijkstra.run\_example()}
        \label{fig:example}
    \end{figure}

    \section{Erdős Rényi random graph}
    The random graphs are made using the Erdős Rényi method. It takes $k$ edges from all possible edges given the node length.
    The edges are always taken to be bidirectional. An example graph generated using the Erdős Rényi method is given in the next figure. 
    Not all nodes are shown in the figure, since not all nodes are connected.
    \begin{figure}[H]
        \centering
        \includesvg[width = 0.6\textwidth]{ER_example.svg}
        \caption{Visualization of a ER random graph with $k=6$ and $N=5$ - output graph of Dijkstra.ER()}
        \label{fig:ER_example}
    \end{figure}

    \section{Performance analysis}
    The performance analysis is done by taking the ER graphs for different sample sizes with the same connectivity. The time complexity is $O(N^2)$ as can be clearly seen in the plot. This is because the implementation is not done using a priority cue.
    \begin{figure}[H]
        \centering
        \includesvg[width = 0.6\textwidth]{time_complexity.svg}
        \caption{Time complexity of the Dijkstra implementation - output of Dijkstra.performance\_analysis()}
        \label{fig:ER_example}
    \end{figure}


    \section{Code}
    \label{section:code}
    The code is also provided seperately and is on \href{https://github.com/SjdTl/DCN-Bonus-Assignment-2.git}{\color{blue}github}.
    \inputminted[breaklines=true]{python3}{Dijkstra.py}
\end{document}